% ==============================================================================
%
%                                    DVG303
%                  Objektorienterad design och programmering
%                                Laboration #1
%
% Author:   Jonas Sjöberg
%           Högskolan i Gävle
%           tel12jsg@student.hig.se
%           https://github.com/jonasjberg
%
% License:  Creative Commons Attribution-NonCommercial-ShareAlike 4.0
%           International.  See LICENSE.md for full licensing information.
% ==============================================================================

\section{Uppgift 1}\label{sec:uppg01}

\subsection{}\label{sec:uppg1a}
\subsubsection*{Frågeställning}
Modellera klasserna som representerar konkreta geometriska figurer som ska
kunna hanteras i programmet och därför ska ingå i programmets datamodell:
Punkt, linje, triangel, cirkel, rektangel. 
\par Beskriv klasserna på ett liknande sätt som i kursboken, kap 2.2, på sida 65 -
tredje upplaga.  Skapa dessutom ett UML-klassdiagram för modellen.
\par Motivera era beslut: Varför skapade ni modellen som den är? Varför har
klasserna de attribut och operationer så som ni valde?

\subsubsection*{Lösning}
\begin{figure}[htbp]
\centering
\includegraphics[width=\linewidth]{diagram/uppgift2-icons.eps}
\caption{Uppgift~\ref{sec:uppg1a}: UML-diagram för geometriska figurer}
\label{fig:uppg1a}
\end{figure}


\subsection{}\label{uppg1b}
\subsubsection*{Frågeställning}
Implementera klasserna som ingår i klassdiagrammet och skapa JUnit-tests för
att testa koden!

\subsubsection*{Lösning}
Se bifogad källkod, klasserna finns i paketen
\texttt{se.hig.oodp.lab.model.simplefigure} och
\texttt{se.hig.oodp.lab.model.figure}.



\subsection{}\label{uppg1c}
\subsubsection*{Frågeställning}
Vilken relation ser ni mellan klassen Vertex2D och figurklasserna resp. mellan
instanser av Vertex2D och instanser av figurklasserna? På vilket sätt 
återspeglas relationen i klassdiagrammet? Ge en förklaring!

\subsubsection*{Lösning}
Klassen \texttt{Vertex2D} är en del av figurklasserna. Figurklasserna består
av minst en \texttt{Vertex2D} som utgör figurens mittpunkt. Figurklasserna
har en varsin lista där ett godtyckligt antal \texttt{Vertex2D}-objekt kan
lagras, beroende på vilken figurklass det handlar om.


%\subsubsubsection{Movable.java}
%\javacode{src/se/hig/oodp/lab/model/component/Movable.java}
%\caption{Interfacet Movable}
%\label{src:movable}

% Se bifogad källkod.


%% Screenshots med Bash, terminalfönsterstorlek 90x40
%\subsection{Skärmdump}
%Se Figur~\ref{fig:uppg01-screenshot} för skärmdump på körning av koden i
%Sektion~\ref{src:uppg01} och Sektion~\ref{src:userinputfilter}.
%
%\begin{figure}[htbp]
%\centering
%\includegraphics[width=\linewidth]{img/01.png}
%\caption{Körning av koden till Uppgift~\ref{sec:uppg01}}
%\label{fig:uppg01-screenshot}
%\end{figure}

